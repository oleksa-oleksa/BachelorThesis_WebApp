% !TeX spellcheck = de
\chapter{Einleitung}
\label{sec:intro}
\section{Motivation und Aufgabestellung}
\label{sec:intro:motivation}
Die vorliegende Arbeit beschäftigt sich mit der Einwicklung einer Datenbank-Applikation für das PSE-Labor\footnote{https://www.http://labor.beuth-hochschule.de/pse} (Labor für Pervasive Systems Engineering), das sich an der Beuth Hoschschule für Technik Berlin befindet und seit fast 5 Jahre ein wichtiger Teil des Studiums im Studiengänge Technische Informatik\footnote{https://www.beuth-hochschule.de/b-ti} und Medieninformatik ist. Während im PSE-Labor stattfindenden Übungsveranstaltungen im Studiengang Technische Informatik werden vorhandene im PSE-Labor die Raspberry Pi Minicomputers (kurz: Raspi)  an die Studierenden verliehen. Zu Beginn einer Laborübung werden die Raspi Boards den Studierenden vom Lehrkraft, der die Übung betreut, übergeben und am Ende der Laborübung zurückgezogen. Aus 16 vorhandenen im Labor Raspis, die markierte mit den Nummern 12-16 von Studierenden nach Hause (home-loan) genommen werden können.\\\\
Nachweislich ist das Vorgehen oft mit Reihe von Problemen verknüpft, die sich jedes Semester und fast jedes Mal wiederholen. Die folgenden Problemen wurden von Mitarbeitern des Labors bereits festgestellt und regelmäßig verlangsamen den Prozess der Verleihung und Übungsführung: 
\begin{itemize}
	\item Studierende kennen ihre am Semesteranfang zugewiesene Gruppennummer auch nach mehreren Wochen nicht und geben den Lehrkraft einen Board mit einer falschen Registriernummer, der einer anderen Gruppe früher zugewiesen wurde und nur von der zugewiesenen Gruppe benutzt werden darf. 
	\item Studierende versuchen  einen Board nach Hause auszuleihen, der zu den Lab-Boards gehört und nur im Labor während der Übungszeit verliert werden darf. Außerplanmäßig von Studierenden darf Lab-Board nicht ausgeliehen und auch mit nach Hause (home-loan) nicht genommen werden.
	\item Es gibt ein Verwaltungsaufwand für die ausleihbaren Home-Boards, die von den Studierenden für jeweils eine Woche mit nach Hause genommen werden können. Die Mitarbeiter müssen handlich die Studentenname, Matrikelnummer, Board und Zeit am Zettel registrieren und in einer Woche überprüfen, ob alle ausgeliehenen Boards pünktlich ins Labor zurückgekommen sind. 
	\item Erfahrungsgemäß können Studierende nach Ablauf der Frist ein Ausleihgerät in einem sehr üblen Zustand der Verschmutzung oder Zerstörung zurückgeben, dass es besteht eine Notwendigkeit den Zustand des Gerätes stets zu kontrollieren, damit es immer bekannt wird, zum welchen Zeitraum Raspi Board zum letzten Mal funktionsfähig war und von wem ausgeliehen wurde.  
	\item Falls gilt ein Raspi Board als verloren, es sollte eine Möglichkeit geben, alle vorherigen Ausleihen anzuschauen und festzustellen, von welchem Studierende es ausgeliehen und nicht zurückgegeben wurde. Mit den Zettelchen, auf denen einen Name von Studierende und eine Board Nummer gemerkt werden, ist es zu aufwändig nachvollziehen.	
\end{itemize}

Somit ist schlusszufolgern, dass eine Notwendigkeit das Verleihprozedere für die Loan-Boards (Lab und Home) mit modernen Mitteln der Technischen Informatik zu lösen schon dringend besteht und eine lohnenswerte Aufgabe für zukünftige Abschlussarbeit ist. 

\section{Technische Basis und Themengebiet}
\label{sec:intro:themengebiet}
Obwohl das Thema der Bachelorarbeit "Entwicklung eine Datenbank-Applikation" lautet, ist es keine einzige Aufgabe nur eine Datenbank zur Ausleihverwaltung zu schrieben ist. Es lässt sich die folgenden Struktur definieren, indem drei verteilte Teilen zu entwickeln ist.\\\\
Erstens wird an einem uComputer ein sogenannten Register-Client realisiert. Dafür ein RFID-Leser an uComputer angeschlossen wird. Register-Klient ist neben dem Eingangstür eines kleinen Lagerraums des PSE-Labors zu platzieren ist, wo Raspi-Boards aufbewahrt werden. Es ist geplant, dass Studierende einen Board selbst aus dem Fach nehmen könnte und dann mit Hilfe des Register-Clients den genommenen Board auf sich oder seine Gruppe registrieren lassen. Der Register-Client hat selbst keinen Zugriff zur Datenbank und sollte nur die abgelesene Daten von der Smartcard der Studierende zur Server schicken. \\\\
Zweitens ist ein Display-Klient zu entwickeln, der den Studierenden es zulässt, die Begrüßung des System und eine Beschreibung die für Ausleihe notwendigen Schritten zu sehen. Es sollte in einem Browser-Fenster die aktuelle Server-Kommunikation und Auskunft angezeigt wird, ob die Ausleihe gelang oder ein Fehler aufgetreten war. Ein Android/iOS-Tablett ist eine gute Alternative für die technische Realisierung, da es die Kommunikation zwischen den Mensch und das System leicht und ohne erweiterte Hinweise zulässt. 

Drittens ist ein Web-Server für die Datenbank-Applikation schließlich zu implementieren. Er umfasst alle Datensätze über die vorhandenen im Labor Raspi-Boards, registrierten zum Kurs Studierenden und die abgewickelten Leihvorgänge.  Web-Server wird mit einem Web-Framework Django erstellt.  Django verfügt nun über die Funktionalität und Datenbasis, um die dafür erforderlichen Aktionen durchzuführen. Als Web-Framework bietet Django eine Reihe von Komponenten und Funktionen (Benutzerauthentifizierung, Hochladen von Dateien, Umgang mit Daten usw.), die bei jeder Webanwendungen benötigt werden. Mit einem Web-Framework muss ein Entwickler keine Zeit damit verschwenden, denselben Code von Grund auf neu jedes Mal zu schreiben.

Wie erfolgt nun die Abwicklung des eigentlichen Leihvorgangs von der Ausleihe bis zur Rückgabe eines Boards?  Zuerst wird eine Studentenkarte am Register-Client ablesen und nachdem sollte einen Name und die Anzahl schon ausgeliehenen Boards am Display-Klient angezeigt werden. Falls der Studierende zum Kurs zugelassen ist, darf dann ein gewünschten Board am Register-Klient abgelesen werden. Es ist möglich, dass zu den schon ausgeliehenen Lab-Board noch zusätzlich einen Home-Board nach Hause mitgenommen wird. Das abgelesene Board ist entweder auszuleihen oder zurückzugeben. Es kann sein, dass einem Studierenden die Home-Loan-Absicht eines Boards (12-16) verweigert wird, da in der Vergangenheit schon einmal vom Studierende ein Board in einem inakzeptabel Zustand zurückgeben war und die Ursachen mit den Mitarbeiter des Labor nicht geklären hat.


 
 