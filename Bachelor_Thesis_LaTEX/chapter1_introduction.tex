% !TeX spellcheck = de
\chapter{Einleitung}
\label{sec:intro}
\section{Vorstellung des PSE-Labors}
\label{sec:intro:about_lab}
Im Labor laufen mittlerweile sehr viele von den beiden Mitarbeitern, Andreas, in eigener Regie geführte und nicht Da diese Projekte formal von den offiziellen Labor-Tätigkeiten und -Schwerpunkten des Labors unterschieden werden müssen und diesen auch nicht zugerechnet werden dürfen, reden wir hier von den Projekten

\begin{itemize}
	\item \textbf{Röst-Mahl-Koch-Kaffeemaschine:} Integration von Connectivity und Entwicklung von IoT-Services für Röst-Mahl-Koch-Kaffeemaschine für Bonaverde\footnote{https://www.bonaverde.com/}. 
\end{itemize}  

\section{Motivation und Aufgabestellung}
\label{sec:intro:motivation}
Im Labor laufen mittlerweile sehr viele von den beiden Mitarbeitern, Andreas, in eigener Regie geführte und nicht Da diese Projekte formal von den offiziellen Labor-Tätigkeiten und -Schwerpunkten des Labors unterschieden werden müssen und diesen auch nicht zugerechnet werden dürfen, reden wir hier von den Projekten

Meine Aufgaben waren in folgende Teilbereiche gegliedert:: 
\begin{itemize}
	\item Den Schwellenwert für Noise Click Board berechnen und ihn durch SPI Bus entsprechend programmieren. 
	\subitem * einen automatischen Alarm mit Hilfe von Summer erzeugen.
	\subitem * einen LED anschalten.
\end{itemize}

\section{Technische Basis und Themengebiet}
\label{sec:intro:themengebiet}
Im PSE-Labor laufen mittlerweile sehr viele von den beiden Mitarbeitern, Andreas und Brian, in eigener Regie geführte und nicht. Da diese Projekte formal von den offiziellen Labor-Tätigkeiten und -Schwerpunkten des Labors unterschieden werden müssen und diesen auch nicht zugerechnet werden dürfen.