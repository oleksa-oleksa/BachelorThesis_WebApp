% !TeX spellcheck = en_GB 
\chapter{Introduction}
\label{sec:intro}
\section{About laboratory for Pervasive Systems Engineering}
\label{sec:intro:about_lab}
Eptecon GmbH gehört zu den Berliner Startup-Szene und wurde in dem Jahr 2016 gegründet. Obwohl das Unternehmen kürzlich gegründet wurde, kann man schon feststellen, dass die erste Phase des Lebenszyklus erfolgreich war und Eptecon ein paar Projekte schon abgeschlossen hat. \\

Als ein Startup hat aber Eptecon bisher mehrere Projekte erfolgreich abgeschlossen.
\begin{itemize}
	\item \textbf{Röst-Mahl-Koch-Kaffeemaschine:} Integration von Connectivity und Entwicklung von IoT-Services für Röst-Mahl-Koch-Kaffeemaschine für Bonaverde\footnote{https://www.bonaverde.com/}. Sie nutzt IoT für das Sammeln von Gerätedaten und die Steuerung des Kaffeeherstellungsprozesses. Dies ermöglicht Fernwartung, Neubestellung und Bezahlung von Kaffeebohnen sowie neuartige Benutzererfahrung durch Integration eines Messenger Bots.
	\item \textbf{iPhone Add-on für EKG Messung:} Technologieberatung und Vorbereitung der Massenproduktion eines Add-ons für einfache Messung von Elektrokardiogrammen (EKG) für CardioQvark\footnote{http://www.cardioqvark.ru}. Das EKG-Mess-Add-on ermöglicht die Erfassung von EKG mit nur zwei Fingern. Das aufgezeichnete EKG wird dann zur automatischen Analyse und Verarbeitung an die Cloud geschickt. Die Ergebnisse werden in einer iOS-Anwendung angezeigt und können sehr einfach mit dem zuständigen Arzt geteilt werden. 
	\item \textbf{Wearable UV-Sensor für Sonnenbrandprävention:} Technologiebewertung und Systemdesign eines tragbaren Produktes zur Messung der Sonneneinstrahlung für UVizr\footnote{http://www.uvisio.com}. Ein kleines, tragbares Accessoire mit UV-Sensor warnt seinen Besitzer vor möglichem Sonnenbrand. Eine Smartphone-Anwendung kommuniziert die UV-Empfindlichkeit der Haut des Benutzers an den Sensor und erhält UV-Messdaten zur weiteren Verarbeitung über Bluetooth. 
	\item \textbf{Iot-Plattform für LED-Beleuchtung:} Systemdesign und Entwicklung einer modularen Plattform für die Verbindung von intelligenter Beleuchtung mit dem Internet der Dinge für lumilabs.\footnote{http://www.lumilabs.de}. Die Plattform kombiniert Benutzer- und Sensor-basierte LED-Leuchtensteuerung mit drahtloser und drahtgebundener Kommunikation sowie Datenanalytik. Es lässt sich problemlos in nahezu jede Leuchte integrieren und ermöglicht, neben der Lichtsteuerung, auch das Erfassenen und Bereitstellen von Umgebungsdaten. 
\end{itemize}  
