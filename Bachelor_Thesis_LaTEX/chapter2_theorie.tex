% !TeX spellcheck = de_DE
\chapter{Der theoretische Hintergrund}
\label{sec:theorie}
Als schon im Abschnitt "Einleitung" erwähnt wurde, bei der Aufgabestellung es um eine Entwicklung einer Webanwendung geht. Die zu realisierende Webanwendung basiert sich, wie die meisten Webanwendungen, auf einer Client-Server-Architektur, wobei der Client Informationen eingibt, während der Server die eingegebene Informationen empfängt, bearbeitet und speichert.
Eine Webanwendung ist ein Computerprogramm, das eine bestimmte Funktion unter Verwendung eines Webbrowsers als Client ausführt. Die Webanwendung kann so einfach wie ein Kontaktformular auf einer Website oder so komplex wie eine Textverarbeitungs- oder Bildbearbeitungsprogramm sein, die Sie auf Ihr Computer im Browser

\section{Über Raspberry Pi Board und OS}
\label{sec:theorie:raspberry}

\section{Kontaktlose Chipkartentechnik MIFARE}
\label{sec:theorie:mifare}

\section{Sender-Empfänger-System mit RFID}
\label{sec:theorie:rfid}

\section{Datenbanken mit Python und SQLite}
\label{sec:theorie:db}

\section{HTTP für Design der verteilten Systeme}
\label{sec:theorie:http}

\section{Django Framework}
\label{sec:theorie:about_django}

\section{API}
\label{sec:theorie:api}

\section{Endliche Zustandsmaschine}
\label{sec:theorie:fsm}

\section{Clientseitiges JavaScript}
\label{sec:theorie:js}




