% !TeX spellcheck = de_DE 
\chapter{Implementierung}
In meiner Abschlussarbeit präsentiere ich die praktische Lösung für das PSE-Labor der Beuth Hochschule für Technik Berlin. Die gesamte Aufgabe lässt sich in drei Bestandteile unterteilen: Register-Client, Server und Display-Client. Erstens wird Register-Client implementiert, damit wird RFID Leser am Raspberry Pi Mikrocomputer angeschlossen, alle Treiber installiert und auf Python Programmierung Sprache die Software geschrieben, die die ständige Überwachung des empfangenden von RFID Leser Daten zulässt und die Verbindung mit dem Server zulässt. Falls die empfangene Daten korrekt sind, d.h.  eine richtige MIFARE Studentenkarte oder einen richtigen RFID-Transponder abgelesen wurde, schickt die Software die abgelesene Daten zum Server ab. Der Server ist der zweite Bestandteil der Abschlussarbeit und wird mit Hilfe Django Framework, Django Finite State Machine auf Python Programming Sprache implementiert. Server enthält die Datenbank mit die Datensätzen über die alle im PSE-Labor vorhandenen ausleihenden Boards, die zum Modul im laufenden Semester registrierten Studenten und  geschehenen Ausleihe/Rückgabe-Vorgänge. Es wird von Server überprüft, ob eine von Register-Client abgelesene Studentenkarte einem zugelassenen für die Ausleihe Student gehört und die entsprechenden Information auf Display-Client geschickt. Es wird auch von Server bestätigt, ob für die Ausleihe/Rückgabe neben dem RFID-Leser gehaltenen Raspi Board dem Student ausgeliehen/vom Student zurückgegeben werden darf. Drittens wird der Display-Client als dynamische HTML-Seite realisiert, die eine Verbindung zum Server Mithilfe des HTTP-Protokolls und eingebauten im Browser Kommunikationsmittel die asynchrone Nachrichten zu schicken, bereitstellt. Für die dynamische Aktualisierung des Inhalts der Webseite und einen Zugang zum asynchronen HTTP-Client wird jQuery benutzt.
\section{Register-Client}
\label{sec:register_client}
Das folgende Kapitel beschäftigt sich mit der Implementierung des Register-Client auf Raspberry Pi Board mit angeschlossenen RFID-Leser. Dieser Teil der verteilte System lässt sich wie folgendes unterteilen. Zuerst wurde das Betriebssystem Raspbian auf Board zum Leben gebracht und dann die alle notwendigen für RFID-Leser Treiber installiert. Nach dem der RFID-Leser funktionieren angefangen und die Daten von RFID-Transponder abgelesen hat, wurde die nächste Herausforderung gelöst: die Struktur die zu empfangenen Daten wurde verstanden, richtig bearbeitet, eine JSON-Datei erstellt und durch die HTTP-Protokoll dem Server geliefert. 

\section{Register-Client}
\label{sec:register_client:install}

\section{Server}
\label{sec:server}


\section{Display-Client}
\label{sec:display_client}
