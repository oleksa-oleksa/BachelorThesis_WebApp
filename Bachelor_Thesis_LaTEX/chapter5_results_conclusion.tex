% !TeX spellcheck = de_DE
\chapter{Analyse}
\label{sec:analyse}
An dieser Stelle muss es gesagt werden, dass die RFID-Tags bestimmte Nachteile haben, die wurden zwischen den Mitarbeiter und Autorin der Abschlussarbeit diskutiert. Es wurde jedoch die Entscheidung getroffen, dass RFID-Tags für die vorgesehenen Zwecken des Verfolgen des Raspi-Boards (welchen Raspi-Board von welchem Student am welchen Tag ausgeliehen wurde und bis zum welchen Tag zurückgegeben muss) die Sicherheitserwartungen der PSE-Labor Mitarbeiter erfühlen. RFID-Tags sind aus mehreren Gründen im Vergleich nicht ideal. Da ein RFID-Tag nicht zwischen Lesegeräten unterscheiden kann, können die Informationen von fast jedem gelesen werden, sobald sie die ursprüngliche Lieferkette verlassen haben. Weil RFID-Lesegeräte so tragbar sind und die Reichweite einiger Tags so groß ist, können Betrüger Informationen sammeln, auf die sie sonst keinen Zugriff hätten. Dies bedeutet, dass jeder ohne Wissen einer Person potenziell sensible Informationen sammeln kann. Diese Nachteile der RFID-Tags wurden vernachlässigt, da sowohl Studentenkarte als auch geklebte auf den Raspi-Boards RFID-Tags keine sensible Information behalten. Es gibt trotzdem ein Gefahr, dass entweder die Studentenkarte geklont von einem Täter wird, um sich für einen Student ausgeben und ein Raspi-Board stehlen zu können, oder ein Raspi-Board geklont von einem Täter wird, um ein Datensatz in der Datenbank zu erzeugen, dass schon ausgeliehenen Board quasi zurückgegeben wurde, obwohl in der Realität den Raspi-Board nie zurückgegeben wurde. Gegen das Klonen des Raspi-Tags wurde es besprochen, dass in der Zukunft im PSE-Labor im Schrank mit den Raspi-Boards jeden Platz für jeden entsprechenden Raspi-Board mit einem Gewichtssensor ausgerüstet werden wird und acaLoan-System auf der Erscheinung des bestimmten Gewicht erwarten wird. Dies ist aber nicht der Teil bestehenden Abschlussarbeit und von Mitarbeiter des PSE-Labor als eine spannende Aufgabe für die andere Abschlussarbeit vorgesehen ist. Gegen das Klonen des Studentenkarte wurde es zuerst entschieden, dass ein bestehenden Zugang zu einem Schrank mit Raspi-Boards entlang die beide Arbeitstischen der Mitarbeiter des PSE-Labor eine bestimmte Sicherheit gewährleisten könnte. Nach anderen Lösungen wird es weiter noch diskutiert und es liegt außer den Rahmen der bestehenden Abschlussarbeit. 

Zusammenfassend lässt sich sagen, dass die neue Studentenkarte, die an der Beuth Hochschule ab Sommersemester 2018 verwendet wurden, sind eine zuverlässige und zeitgemäße Lösung. Die Karte beinhaltet keine elektronischen persönlichen Daten der Studierenden und die Campus-Automaten alle persönlichen Daten anhand eines Pseudonyms online abrufen müssen (d.h. liegen in den Automaten auch keine persönlichen Daten vor). Sodass im Fall des Verlusts die persönliche Daten von den Unberechtigte nicht ausgelesen werden können \cite{website:12}. Das stand im Fokus der Entscheidung, eine Studentenkarte als einzige elektronischer Identifizierungsmittel beim Ausleihe/Rückgabevorgänge im PSE-Laboz zu benutzen.

\chapter{Zusammenfassung}
\label{sec:results}
Das zu Beginn der Arbeit gesetzte Ziel der Entwicklung des acaLoan-System wurde mithilfe des Django Framework erfolgreich erreicht. Die Implementierung alle Bestandteilen und ein Zusammenspiel unter der Verwendung des Entwicklungsservers wurde erfolgreich den Mitarbeitern des PSE-Labors präsentiert. Mit der acaLoan-System werden die Studierende in die Lage versetzt, einen Raspi-Board selbständig im PSE-Labor für die Übungen auszuleihen und später zurückzugeben. Damit wird ein Arbeitszeitverbrauch für die Verwaltung der Bordstandortbestimmung reduziert und die Mitarbeitern können sich auf weitere neuen geistliche wissenschaftlichen Herausforderungen konzentrieren und damit in der Weiterentwicklung des acaLab-Projekte hineinbringen.  

In der zu realisierenden Abschlussarbeit ist auch notwendig einen weiteren Teil namens Register-Client zu entwickeln, an dem ein RFID-Leser angeschlossen wird. Der Register-Client verfügt selbst über keinen Datenbank und darf nur die abgelesene Daten dem Server schicken. Es geht um eine Simplex-Verbindung, da ein Nachrichtenverkehr asymmetrisch ist, weil der Register-Client keine Daten vom Server zurückbekommen darf und über den erfolgreiche oder gescheiterte Leihvorgang nicht wissen muss. Für die Implementierung des Register-Clients wird uComputer Raspberry Pi benutzt, der möglicherweise nicht der einzige Single-Board-Computer (SBC) auf dem Markt ist, aber bei weitem der beliebteste und schon zur Verfügung im PSE-Labor steht und ergänzend nicht geliefert werden muss. 

Die Webanwendung könnte theoretisch nur aus dem Client-Teil bestehen, wenn Benutzerdaten nicht länger als eine Sitzung gespeichert werden müssen. Dies aber ist nicht der Fall der Abschlussarbeit, da die Studentenkarten und der Verlauf des Verleihablaufs mindestens für ein laufenden Semester gespeichert werden muss, damit die Mitarbeiter des PSE-Labor immer eine Zugang zu allen gespeicherten vorherigen Leihvorgangs von der Ausleihe bis zur Rückgabe eines Boards. Es ist vorgesehen, dass am Ende des Semester nach dem letzte Rückgabe eines Boards die Datensätzen des zu Ende gegangen Semesters gelöscht wird. 

Daraus lässt sich die Schlussfolgerung ziehen, dass eine Webanwendung für einen Endnutzer wie eine Website aussieht, auf der  die Webseiten mit teilweise oder vollständig nicht formatiertem Inhalt sich befinden. Die Endfertigung des Inhalts findet nur dann statt, nachdem ein Website-Besucher die Seite vom Webserver angefordert hat. 

Im diesen Kapitel wird die Bedeutung der RESTful-API erklärt, das ein Paradigma für die Softwarearchitektur von verteilten Systemen ist und von dessen die Kommunikation zwischen zwei Bestandteilen des acaLoan-System erledigt wird. Client-Server Kommunikation geschieht über HTTP-Protokoll, das für Hypertext Transfer Protokoll steht und dient zum Verwaltung der Übermittlung eines Dokuments durch einen Webserver an einen Webbrowser. HTTP wird auch zum Übertragen von XML-Dateien, VoiceXML, WML, Streaming von Video und Audio verwendet. Es verwendet normalerweise Port 80 und as Transportschichtprotokoll - TCP. Das in RFC 1945 (HTTP 1.0 \cite{website:httprfc1945}), 2068 \cite{website:httprfc2068} und 2616 (HTTP 1.1) definierte WWW-Protokoll, mit dem HTML-Dokumente über das Internet von Knoten zu Knoten gesendet werden können. HTTP unterstützt die dauerhafte Übertragung (Übertragung mehrerer Objekte). Nicht persistente Verbindungen (Übertragung eines Webdokumentobjekts pro Sitzung zwischen Client und Server) sowie zwei Methoden zur Benutzeridentifizierung: Autorisierungs- und Cookie-Objekte (Dateien). Dies ist ein zustandsloses Protokoll, in dem keine Benutzersitzungsinformationen gespeichert werden. Jede Datenübertragung wird als neue Sitzung für die Kommunikation zwischen verteilten Informationssystemen betrachtet \cite[p.62]{shklar:webapplication}.

