% !TeX spellcheck = de_DE
\chapter{Systemdesign}
\label{sec:design}
Nachdem die Aufgabestellung des Projekts wurde von PSE-Labor veröffentlicht und mit Autorin der Abschlussarbeit besprochen und alle Unklarheiten geklärt, die erste Wahl der Hardware gemacht und die Hardware bestellt wurde und während es auf die Lieferung in das PSE-Labor gewartet wurde, wurde der erste notwendige Schritt angefangen: Systemdesign. Im Rahmen des Systemdesigns werden die Struktur und Zusammenhänge der Elemente des zu entwickelnden Systems untersucht. Das Ergebnis dieser Untersuchung enthält genügend Informationen, um das System zu implementieren. Zuerst wurden die User Stories erstellt. Die sind keine detaillierte Beschreibung der Anforderungen (d.h. was das System tun sollte), sondern eine diskutierte Darstellung der Absicht (Endbenutzer muss/will so etwas tun). Die sind kurz und leicht zu lesen sowohl für Entwicklerin (Autorin) als auch für Stakeholder (mindestens PSE-Labor Mitarbeiter) verständlich. 

Es ist mithilfe der User Stories zu klären, was die zu entwickelnde Datenbank leisten soll. Es wird auf die Frage konzentriert, welche Daten in der Datenbank gespeichert werden sollen. Dazu wird zunächst  die existierende und vorgesehenen Ausleihe-/Rückgabelvorgängen betrachtet und geklärt, welche Informationsobjekte (Personen, Sachen) von Interesse sind und wie diese miteinander in Beziehung stehen. Während der Analysephase wurde auch Sequenzdiagramm erzeugt, die einfach die Interaktionen zwischen Objekten in einer sequentiellen Reihenfolge zeigt, d.h. die Reihenfolge, in der diese Wechselwirkungen stattfinden. Die Sequenzdiagramm wurde als nächstes als eine Basis für das Design des Zustandsmaschine verwendet.  

Für das Systemarchitektur wird das UML-Klassendiagramm entwickelt, das einen Überblick über ein Softwaresystem bietet, indem Klassen, Attribute, Operationen und deren Beziehungen angezeigt werden. Dieses Diagramm enthält den Klassennamen, die Attribute und die Operation in separaten festgelegten Fächern. In der Entwurfsphase wird es festgelegt, welche Klassen das System benötigt wird. Eine Klasse ist eine Vorlage zum Erstellen von Objekten, die Initialisierung von Variablenfeldern und Implementierung des Verhaltens von Feldern und Methoden bereitstellt. Die festgestellte Klassen werden weiter nicht wie üblichen Python-Klassen implementieren, jedoch wie eine Django Modelle direkt zum Erzeugen der Datenbank verwendet. 





\section{Anforderungen und User Stories}
\label{sec:design:user_stories}
Die Softwareentwicklung beginnt normalerweise mit einer Beschreibung der Bedürfnisse und ihrer Analyse. Je genauer und korrekter die Beschreibung der Softwareanforderungen und deren Analyse ist, desto einfacher ist es, alle nachfolgenden Schritte abzuschließen. Das Hauptproblem in dieser Phase ist der Unterschied in den Ansichten des Kunden (in dem Fall der vorliegenden Abschlussarbeit sind die Kunden die PSE-Labor Mitarbeiter) und des Entwicklers (die Autorin der Abschlussarbeit in diesem Fall). Am Anfangs des Abschlussprojekts wurde eine detaillierte Aufgabestellung von PSE-Mitarbeitern geschrieben und auf Beuth Website für die interessierende in der Abschlussarbeiten Studierende veröffentlicht\cite{website:17}. Nachdem die Autorin der Abschlussarbeit auf die Herausforderung achtgegeben hat und mit den PSE-Mitarbeitern bezüglich der Aufgabe kontaktiert, wurde es eine Reihe von Treffen gegeben, während deren haben sich die Kunden (PSE-Mitarbeiter) und die Entwicklerin (die Autorin) sich über die Bedürfnisse der Kunden und die Funktionalität der zu einwickelnde Software geeignet. Es wurde auch die Entscheidung getroffen, mit welchen Hardware ist die Aufgabestellung zu implementieren. Jedoch während der Entwicklung der Register-Client (der einer von drei Bestandsteilen der Software, an dem ein RFID-Leser angeschlossen werden muss), wurde schließlich den RFID-Leser gewechselt. Der Fall ist im Kapitel \ref{sec:register_client:install_rfid} nachzulesen.

Die Analyse der Anforderungen hat sich mit den User Stories angefangen. Eine User Story ist eine Anforderung, die aus der Perspektive eines Endbenutzerziels ausgedrückt wird Sie nehmen keine großen, umständlichen Dokumente auf, sondern sind in Listen gesammelt, die einfacher zu organisieren und neu zu ordnen sind, wenn neue Informationen eintreffen. User Stories werden nicht zu Beginn des Projekts detailliert beschrieben, sondern bereits "just in time" detaillierter entwickelt, um zu frühe Sicherheiten, Verzögerungen bei der Entwicklung, Anhäufung von Anforderungen und eine zu eingeschränkte Formulierung der Lösung zu vermeiden. Sie erfordern wenig oder keine Wartung und können nach der Implementierung sicher abgebrochen werden. Aus der Sicherheitsrunden falls der Festplatte der Rechner der Autorin kaputt geht oder irgendwie anders gehen die Daten die zu einzuwickelnden Software verloren, wird Versionskontrollsystemen namens Git benutzt. Anstatt nur einen einzigen Platz für den vollständigen Versionsverlauf der Software zu haben, wird in Git die Arbeitskopie jedes Entwicklers des Codes auch ein Repository. Das kann den vollständigen Verlauf aller Änderungen enthalten. Während des Corona-Semester, wann den Zugang zum Labor gesperrt wurde und die weitere Kommunikation mit den PSE-Labor Mitarbeitern und Autorin schließlich über Internet geschah, hat Git es auch erlaubt, den Kunden der Software den Fortschritt der Entwicklung zu sehen. 

Seit Oktober 2016 bietet GitHub die Möglichkeit, GitHub-Probleme, Pull-Anfragen und Notizen mit Projekten zu verfolgen. Mit GitHub-Projekten können Boards im Kanban-Stil für die Verwaltung der Arbeit verwendet und separate Code-Repositorys durchschnitten werden. 


Während die meisten neuen Funktionen mithilfe der User Stories aus Anwendersicht definiert werden sollten, ist dies nicht immer machbar oder sogar hilfreich, wenn es zu Sicherheitsfunktionen oder Infrastrukturanforderungen kommt, die nicht kundenorientiert sind. Die Anforderungen sind in der Regel mehr detailliert und das Schreiben dauert länger. Diese gehen oft technisch auf die Funktionsweise der Software ein. Diese Details leiten das Entwicklungsteam dann beim Erstellen eines neuen Features oder einer neuen Funktionalität.

% + Sequencdiagramm
Sequenzdiagramme beschreiben, wie und in welcher Reihenfolge die Objekte in einem System funktionieren. Diese Diagramme wird häufig verwendet, um Anforderungen an neue und vorhandene Systeme zu dokumentieren und zu verstehen.

\section{Systemarchitektur}
\label{sec:design:arch}
Das Klassendiagramm definiert die Objekttypen im System und die verschiedenen Arten von Beziehungen, die zwischen ihnen bestehen. Es bietet eine allgemeine Ansicht einer Anwendung. Diese Modellierungsmethode kann mit fast allen objektorientierten Methoden ausgeführt werden. Eine Klasse kann sich auf eine andere Klasse beziehen. Eine Klasse kann ihre Objekte haben oder von anderen Klassen erben.

\section{Endliche Zustandsmaschine}
\label{sec:design:fsm}
