% !TeX spellcheck = en_GB 
\chapter{Application}

\section{Usage in laboratory}
\label{sec:usage}
Der Standard legt das physikalische Layout der mikroBus-Pinout-Verbindung, die verwendeten Kommunikations- und Stromversorgungspins auf dem Mainboard fest. Der Zweck von mikroBUS ist es, eine einfache Erweiterbarkeit der Hardware mit einer großen Anzahl von standardisierten kompakten Zusatzboards zu ermöglichen, von denen jede einen einzelnen Sensor, Display, Encoder oder Motortreiber, eine integrierte Schaltung hat. Der von MikroElektronika entwickelte mikroBUS ist ein offener Standard - jeder kann mikroBUS in seinem Hardwaredesign implementieren. Die Abbildung\footnote{https://download.mikroe.com/documents/standards/mikrobus/mikrobus-standard-specification-v200.pdf} \ref{fig:mikrobus} zeigt die Pinout Spezifikation des Herstellers, die man entsprechend ändern kann und die neue Verbindungen für den eigenen Projekt feststellen. Wenn ein Modul eine Schnittstelle verwendet, die bereits auf mikroBUS vorhanden ist, benutzt man diese exakten Pins und markiert diese entsprechend. Wenn ein Pin nicht verwendet wird, sollte er als NC (für "Not Connected") markiert sein. 